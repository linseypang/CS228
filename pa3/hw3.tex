\def\s{\section*}
\def\ss{\subsection*}
\def\sp{\hspace{3 mm}}
\def\ind{\!\perp\!\!\!\perp}
\def\m{\mid}
\def\nl{ \\ = \sp &}
\newcommand{\eq}[1]{\begin{align*}&{#1}\end{align*}}
\renewcommand{\ss}[1]{\subsection*{#1}}
\renewcommand{\P}[1]{\s{Problem #1}}
\def\ra{\rightarrow}
\def\la{\leftarrow}
\def \pb{\newline\newline}
\newcommand{\pic}[2]{\begin{figure}[H]
  \includegraphics[width=\linewidth]{#2}
  \caption{#1}
  \label{fig:net}
\end{figure}}

\documentclass[10pt,a4paper]{article}
\usepackage[utf8]{inputenc}
\usepackage{amsmath}
\usepackage{parskip}
\usepackage{amsfonts}
\usepackage{lmodern}
\usepackage{float}
\usepackage{graphicx}
\usepackage{amssymb}
\author{Hugh Zhang}
\title{CS 228 Problem Set 1}
\date{\today}
\begin{document}
\maketitle

\P 1

Let the structure be a two node binary Bayes net with one edge from A to B. Let $P(A=1) = 0.51, P(B=1 \mid A=1) = .5, P(B=1 \mid A=0) = .99$

Then, $P(B=1) = 0.51*.5 + 0.49*.99 = .7401$, so the most likely outcome based on marginals without looking at conditionals is $A=1, B=1$. However, the $P(A=1,B=1) = .51*.5 = .255$. This is not as likely as the most likely outcome, A=0,B=1 which has $P(A=0,B=1) = .49*.99 = .4851$.

\P 2

Basic message passing is $nd^2$ for a chain. Labeling the nodes from A to Y WLOG, you can factor
\[
P(A\dots Y) = \phi(A,B) * \phi(B,C) \dots \phi(X,Y)
\]
Z, which is the normalizing constant is just
\begin{align*}
& \sum_A\dots\sum_Y P(A\dots Y) = \sum_A\sum_B\phi(A,B) * \sum_C\phi(B,C) \dots \sum_Y\phi(X,Y)
\end{align*}

Z takes $nd^2$ time to calculate because this is just message passing across the entire array. Note that when we try calculating the marginals for a pair of variables, the variables cut out three parts of the chain. WLOG call them $x_i$ and $x_j$ with $i<j$. For all $x_k$ with $k<i$ or $x_m$ with $m>j$, then you can do basic message propogation to calculate each chain for $nd^2$ using basic message passing. Then, to get the middle part [bottom of the equation listed below], you can choose to extend the left side (WLOG) and calculate the middle chain for each fixed value of $x_i$, and then unite it with the right chain, then normalize to have a complete probability. Thus, since you do message passing d times when you try to marginalize the variable out (once for each value of the variable), it takes $nd^3$ the middle calculation still dominates the run time. To calculate all $n^2$ pairs, the runtime is then $n^3d^3$


Then \begin{align*}
& P(x) = \frac{l(x) \phi(X_i,x_{i+1}) \dots\phi(x_{j-1},X_j)*r(x)}{l(x) \sum_{x_i}\phi(x_i,x_{i+1}) \dots \sum_{x_j}\phi(x_j-1,x_{j})*r(x)}
\end{align*}

\ss{2.2}

Note that $X_i$ and $X_j$ are independent conditioned on $X_{j-1}$
\[\sum_{x_{j-1}} P(X_i,X_{j-1})P(X_j \mid X_{j-1}) = \sum_{x_{j-1}} P(X_i,X_{j-1},X_j) = P(X_i,X_j) 
\]

\ss{2.3}

I claim you can do this in $O(n^2d^3)$ time. First you calculate Z, which takes $O(nd^2)$, which is fine. Then we calculate all the individual marginals $P(x_i)$ for all i. This is also just basic message passing, so its $O(nd^2)$  Then, you calculate all adjacent pairs using our above algorithm. $P(x_1,x_2) \dots P(x_{n-1},x_n)$ This takes $nd^3$ time per pair, and N pairs, so we are at $O(n^2d^3)$.
Then, to calculate all pairs of distance x apart given that you've calcualted all pairs of distance x-1 apart and below, we use the formula above. (This is the inductive step)
\begin{align*}
& P(X_i,X_j) \nl \sum_{x_{j-1}} P(X_i,X_{j-1})P(X_j \mid X_{j-1}) \nl
\sum_{x_{j-1}} P(X_i,X_{j-1}) \frac{P(X_j, X_{j-1})}{P(X_{j-1})}
\end{align*}

We have all of these values, since $j-i-1 = x-1 <x$, and we can use our previously calculated values. There are $d^2$ possible values of $X_i$ and $X_j$, and N possible pairs of distance X. Marginalizing over all values of $X_{j-1}$ adds another factor of d to our calculations for a total of $nd^3$ per inductive step. Since we have n inductive steps, this is a total of $n^2d^3$, exactly what we want.

\P 3

If you change a factor (by changing its value, or by adding an edge) in a clique, check all its neighbors and see if they are affected, AKA the seperation set contains the factor that has been changed. If they are affected, then everything in that entire subtree needs to be updated, since they all depend on that factor in the message passing. If the factor is not in the sep-set, then it gets marginalized out anyways.

\ss{3.2}

Pick any clique that contains X. Once you find the probabiity of all the variables in the clique, you can marginalize out the other variables so that you have P(x). The markov blanket for this, is all the neighboring cliques. If the factor you changed is not inside this markov blanket, you don't need to do anything since it all gets marginalized out anyways.

If your factor is inside your markov blanket, then you just need to make sure that all the incoming messages to the clique that contains X is correct, since P(C) is just the factor times the product of all the incoming messages with marginalization. If your factor is inside the clique, you just need two passes. One for the outward messages in which the sep-set also contains the factor you changed to make sure your neighbors are correct, then one for the inward messages to make sure you are calibrated. If you are in among the neighbors, just updating the message inward (if the factor is in the sep-set) is sufficient.

\P 4

\ss{4.2.1}

$P(X \mid E)$ is computationally intractable (NP hard to calculate on a general bayes net). Sampling from this distribution is also hard, since you can't calculate $P(X \mid E)$ for any given X easily.

Computing $P(X,E)$ though is doable, since you just walk through the entire Bayes tree and propogate down on your probabilities. Forwardprop

\ss b



\P 5

\ss a

CODE HERE XXXXXXXXXXXXXXXXXXXXXXXXXXXXXXXX

    G.varToFac = [[i] for i in range(M)]

    for i in range(N):
        for var in np.where(H[i]==1)[0]:
            G.varToFac[var].append(i+M)

    G.factor = []
    G.M = M
    G.N = N
    G.p = p

    for i in range(M):

        if yhat[i] == 1:
            vals = [p, 1.-p]
        if yhat[i] == 0:
            vals = [1.-p, p]
        G.factor.append(Factor(None, [i], [2], np.array(vals)))
    for i in range(N):
        scope = np.where(H[i]==1)[0]

        val = np.zeros([2]*scope.size)
        for index, value in np.ndenumerate(val):
            s = 0
            for i in index:
                s += i
            if s%2 == 0:
                val[index] = 1.

        G.factor.append(Factor(None, scope, [2]*scope.size, (val)))

    return G
    
XXXXXXXXXXXXXXXXXXXXXXXXXXXXXXXXXXXXXXXX

Test cases:
ytest1 = [1, 1, 1, 1, 1]
ytest2 = [1, 0, 1, 0, 1]
ytest3 = [0, 0, 0, 0, 0]

If you include the error rate factors:
0, 0, 0.77378094
If you don't: 
0, 0, 1

\ss c

All values essentially 0, as expected. Thus, the code word is properly decoded.

\pic{Plot with the ith bit on the X axis and the probability of it being 1 on the y axis.}{5c.png}

\ss d

It's moderately reliable. For a 0.06 error rate, all but 1 message converged to the correct message after only 50 iterations.

\pic{Error rate of 0.06. X axis is the number of iterations. Y axis is the hamming distance between decoded message and real message}{Point6.png}

\ss e

As the error rate increases, it gets harder and harder for loopy BP to correctly converge onto the correct message. Only 5/10 messages converged to a 0 hamming distance for error rate of 0.08. For the error rate of .1, many didn't even come close and didn't show much sign of going down.

\pic{Error rate of 0.08. X axis is the number of iterations. Y axis is the hamming distance between decoded message and real message}{Point8.png}
\pic{Error rate of 0.1. X axis is the number of iterations. Y axis is the hamming distance between decoded message and real message}{Point1.png}

\ss f

Pretty much decodes picture after 30 iterations. At iteration 0, the picture is non existent since all the incoming messages to variables are initialized uniformly.

\pic{With 0.06 error rate and 0 iteration}{{0.06_0}.png}
\pic{With 0.06 error rate and 1 iteration}{{0.06_1}.png}
\pic{With 0.06 error rate and 2 iteration}{{0.06_2}.png}
\pic{With 0.06 error rate and 3 iteration}{{0.06_3}.png}
\pic{With 0.06 error rate and 5 iteration}{{0.06_5}.png}
\pic{With 0.06 error rate and 10 iteration}{{0.06_10}.png}
\pic{With 0.06 error rate and 20 iteration}{{0.06_20}.png}
\pic{With 0.06 error rate and 30 iteration}{{0.06_30}.png}

\ss g

At iteration 0, the picture is non existent since all the incoming messages to variables are initialized uniformly. Naturally, since there is a higher error rate, the picture looks more blurry at first, but it doesn't really correct to perfection unlike the error rate of 0.06. At some point, the parity checks get so muffled that you can no longer make out what the picture was supposed to be.

\pic{With 0.1 error rate and 0 iteration}{{0.1_0}.png}
\pic{With 0.1 error rate and 1 iteration}{{0.1_1}.png}
\pic{With 0.1 error rate and 2 iteration}{{0.1_2}.png}
\pic{With 0.1 error rate and 3 iteration}{{0.1_3}.png}
\pic{With 0.1 error rate and 5 iteration}{{0.1_5}.png}
\pic{With 0.1 error rate and 10 iteration}{{0.1_10}.png}
\pic{With 0.1 error rate and 20 iteration}{{0.1_20}.png}
\pic{With 0.1 error rate and 30 iteration}{{0.1_30}.png}

\end{document}
